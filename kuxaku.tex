%% LyX 2.3.5.2 created this file.  For more info, see http://www.lyx.org/.
%% Do not edit unless you really know what you are doing.
\documentclass[english]{article}
\renewcommand{\familydefault}{\sfdefault}
\usepackage[T1]{fontenc}
\usepackage[latin9]{inputenc}
\usepackage[a4paper]{geometry}
\geometry{verbose,tmargin=2cm,bmargin=2cm,lmargin=2cm,rmargin=2cm,headheight=2cm,headsep=2cm,footskip=1cm}
\usepackage{textcomp}
\usepackage{graphicx}

\makeatletter
%%%%%%%%%%%%%%%%%%%%%%%%%%%%%% User specified LaTeX commands.
\usepackage{../kuxaku/kuxaku}

\makeatother

\usepackage{babel}
\begin{document}
\title{Solar System}

\maketitle
Positions of Solar System bodies and effective distances between them
for communiation and travel.

\section{Inner System}

The inner system map shows positions of notable objects inside Jupiter's
orbit. Future orbit positions are displayed for some bodies at one
month intervals.
\begin{description}
\item [{Tycho~Station}] The Belt headquarters of Tycho Manufacturing and
Engineering Concern is the largest mobile construction platform in
the Sol System. Fifteen thousand workers and their families live within
Tycho, building megastructures or massive ships far beyond the reach
of a planet.
\end{description}

\subsection{Colonized Asteroids}
\begin{description}
\item [{1~Ceres}] The only dwarf planet in the inner solar system, and
the first asteroid discovered by humanity. Tycho corporation spun
up the asteroid in a dramatic feat of engineering, granting it a gravity
of 0.3 g. Now the most important port of call in the Belt with population
of approximately six million permanent residents
\item [{2~Pallas}] The third largest asteroid in the Asteroid Belt, and
the second asteroid discovered by humanity. It hosts one of the oldest
stations, Pallas Station, in the outer planets, but is also known
for a revolt at its colony. The station itself has a long history
of a refinement station for the mining operations of the Belt. Due
to this legacy, it continues to have its infrastructure maintained
and upgraded, making use of its older equipment as overflow capacity.
\item [{4~Vesta}] The second-most-massive asteroid in the Asteroid Belt
after Ceres. It hosts one of the largest settlements in the outer
planets. At some time during the UN-MCR Cold War, Vesta was the site
of the Vesta Blockade, a confrontation between the UN and MCR that
would delay the Martian terraforming efforts for over a century.
\item [{10~Hygiea}] The fourth largest asteroid in the Asteroid Belt and
somewhat oblong. It hosts Hygeia Station. Like many places in the
Belt, its population suffered from high UN taxes that made survival
expensive and kept the population routinely destitute.
\end{description}

\subsection{Other Notable asteroids}
\begin{description}
\item [{3~Juno}] One of the two largest stony asteroids, along with 15
Eunomia. It is estimated to contain 1\% of the total mass of the Asteroid
Belt. Its orbit has an extreme eccentricity which brings Juno closer
to the Sun at perihelion than Vesta and further out at aphelion than
Ceres.
\item [{5~Astraea}] The fifth asteroid discovered. Physically unremarkable
but notable because after its discovery, thousands of other asteroids
would follow. The discovery of Astraea proved to be the starting point
for the eventual demotion of the four original asteroids (which were
regarded as planets at the time) to their current status.
\item [{6~Hebe}] Large main-belt asteroid, containing around 0.5\% of
the mass of the Belt. This high bulk density means an extremely solid
body that has not been impacted by collisions, which is not typical
of asteroids of its size. In brightness, Hebe is the fifth-brightest
object in the Asteroid Belt.
\end{description}

\begin{description}
\item [{8~Flora}] Large, bright main-belt asteroid. It is the innermost
large asteroid: no asteroid closer to the Sun has a diameter above
25 kilometres or two-elevenths that of Flora itself, and not until
the tiny 149 Medusa was discovered was a single asteroid orbiting
at a closer mean distance known.
\item [{9~Metis}] One of the larger main-belt asteroids. It is composed
of silicates and metallic nickel-iron, and may be the core remnant
of a large asteroid that was destroyed by an ancient collision. Metis
is estimated to contain just under half a percent of the total mass
of the Asteroid Belt.
\end{description}

\begin{description}
\item [{15~Eunomia}] The largest of the stony asteroids. Eunomian family
is the most prominent family in the intermediate asteroid belt and
the 6th-largest family with nearly six thousand known members, or
approximately 1.4\% of all asteroids in the Asteroid Belt.
\item [{16~Psyche}] One of the most massive asteroids in the asteroid
belt. This object is over 200 km in diameter and contains about 1\%
of the mass of the entire asteroid belt. It is thought to be the exposed
iron core of a protoplanet, and is the most massive metallic M-type
asteroid.
\end{description}
\begin{figure}[h]
\includegraphics[width=1\columnwidth]{output/inner}
\end{figure}

\begin{description}
\item [{20~Massalia}] Stony asteroid and the parent body of the Massalia
family located in the inner region of the asteroid belt, approximately
145 kilometers in diameter. The family is fairly young, estimated
to have been created by an impact 150 to 200 million years ago.
\item [{24~Themis}] The largest member of the Themistian family with surface
completely covered in ice. There is also organic compounds in the
form of tholins, high-molecular weight organics found in the outer
solar system, distinguished by a brown or reddish color in optical
spectra.
\end{description}

\begin{description}
\item [{45~Eugenia}] Famed as one of the first asteroids to be found to
have a moon orbiting it, and the first one to been discovered by an
Earth-based telescope. Eugenia I Petit-Prince is the larger (diameter
of 13 km), outer moon. A second, smaller (diameter of 6 km) satellite
orbits closer to Eugenia.
\end{description}

\begin{description}
\item [{52~Europa}] The 6th-largest asteroid in the asteroid belt, having
an average diameter of around 315 km. It is not round but is shaped
like an ellipsoid of approximately 380\texttimes 330\texttimes 250
km. Europa is a very dark carbonaceous C-type, and is the second largest
of this group.
\item [{65~Cybele}] One of the largest asteroids in the Solar System and
is located in the outer asteroid belt. It gives its name to the Cybele
group of asteroids that orbit outward from the Sun from the 2:1 orbital
resonance with Jupiter. The last outpost of an extended asteroid belt.
\item [{87~Sylvia}] The 8th-largest asteroid in the asteroid belt. It
is the parent body of the Sylvia family and member of Cybele group
located beyond the core of the belt. Sylvia was the first asteroid
known to possess more than one moon. They have been named (87) Sylvia
I Romulus and (87) Sylvia II Remus.
\end{description}

\begin{description}
\item [{90~Antiope}] A double asteroid in the outer asteroid belt. It
was found to consist of two almost-equally-sized bodies orbiting each
other. At average diameters of about 88 km and 84 km, both components
are among the 500 largest asteroids.
\item [{107~Camilla}] One of the largest asteroids from the outermost
edge of the asteroid belt, approximately 220 kilometers. It is a member
of the Sylvia family and located within the Cybele group. The X-type
asteroid is a rare trinary asteroid with two minor-planet moons.
\end{description}

\begin{description}
\item [{216~Kleopatra}] A metallic, ham-bone-shaped asteroid and trinary
system orbiting in the central region of the asteroid belt, approximately
138 kilometers in diameter. It is believed that Kleopatra's shape,
rotation, and moons are due to an oblique impact perhaps 100 million
years ago.
\end{description}
\begin{figure}[h]
\includegraphics[width=1\columnwidth]{images/asteroid}
\end{figure}

\begin{description}
\item [{243~Ida}] An asteroid in the Koronis family of the asteroid belt.
It was the second asteroid visited by a spacecraft and the first found
to have a natural satellite. Ida's moon Dactyl is only 1.4 kilometres
in diameter, about 1/20 the size of Ida.
\end{description}

\begin{description}
\item [{511~Davida}] One of the ten most-massive asteroids, and the 7th-largest
asteroid. It is approximately 270--310 km in diameter and comprises
an estimated 1.5\% of the total mass of the Asteroid Belt. It is a
C-type asteroid, which means that it is dark in colouring with a carbonaceous
chondrite composition.
\end{description}

\begin{description}
\item [{588~Achilles}] Large Jupiter trojan from the Greek camp. Archillies
was the first Jupiter trojan to be discovered. The dark D-type asteroid
measures approximately 133 kilometers in diameter which makes it one
of the 10 largest Jupiter trojans.
\item [{617~Patroclus}] A binary Jupiter trojan approximately 140 kilometers
in diameter. It was the second trojan to be discovered and the only
member of the Trojan camp named after a Greek character. The dark
D-type asteroid is also slow rotator and one of the largest Jupiter
trojans.
\item [{624~Hektor}] The largest Jupiter trojan and the namesake of the
Hektor family, with a highly elongated shape equivalent in volume
to a sphere of approximately 225 to 250 kilometers diameter. It has
one small 12-kilometer sized satellite, Skamandrios.
\item [{944~Hidalgo}] A centaur and unusual object on an eccentric, cometary-like
orbit between the asteroid belt and the outer Solar System, approximately
52 kilometers in diameter. It is the first member of the dynamical
class of centaurs ever to be discovered.
\item [{951~Gaspra}] An S-type asteroid that orbits very close to the
inner edge of the asteroid belt. Gaspra was the first asteroid ever
to be closely approached when it was visited by the Galileo spacecraft,
which flew by on its way to Jupiter on 29 October 1990.
\item [{1036~Ganymed}] A stony asteroid on a highly eccentric orbit, classified
as a near-Earth object of the Amor group. With a diameter of 35 kilometers,
Ganymed is the largest of all near-Earth objects. Amor asteroids is
a subgroup of the near-Earth asteroids that approach the orbit of
Earth from beyond, but do not cross.
\end{description}


\section{Outer System}

The outer system map shows positions of giant planets within orbit
of Neptune. Earth, Mars and some asteroid colonies are included for
reference. Future orbit positions are displayed for some objects at
one year intervals.

\begin{figure}[h]
\includegraphics[width=1\columnwidth]{output/outer}
\end{figure}


\subsection{Centaurs}

Centaurs are small Solar System bodies with either a perihelion or
a semi-major axis between those of the outer planets. They generally
have unstable orbits because they cross or have crossed the orbits
of one or more of the giant planets; almost all their orbits have
dynamic lifetimes of only a few million years. Centaurs typically
behave with characteristics of both asteroids and comets.
\begin{description}
\item [{944~Hidalgo}] The first member of the dynamical class of centaurs
ever to be discovered. The dark D-type object has a rotation period
of 10.1 hours and an elongated shape. Its orbit takes it to the inner
edge of the Asteroid Belt and as far out as to the orbit of Saturn.
\item [{2060~Chiron}] Although Chiron was initially called an asteroid
and classified only as a minor planet with the designation \textquotedbl 2060
Chiron\textquotedbl , it was later found to exhibit behavior typical
of a comet. Today it is classified as both a minor planet and a comet.
\item [{5145~Pholus}] An eccentric centaur in the outer Solar System,
approximately 180 kilometers in diameter, that crosses the orbit of
both Saturn and Neptune. The very reddish object has an elongated
shape and a rotation period of 9.98 hours. It was the second centaur
to be discovered.
\item [{7066~Nessus}] A centaur on an eccentric orbit, located beyond
Saturn in the outer Solar System. The dark and reddish minor planet
is elongated and measures approximately 60 kilometers in diameter.
It has a relatively long orbital half-life of about 4.9 million years.
\item [{8405~Asbolus}] A centaur orbiting in the outer Solar System between
the orbits of Jupiter and Neptune. It measures approximately 80 kilometers
in diameter and has a fresh impact crater on its surface, less than
10 million years old.
\item [{10199~Chariklo}] The largest centaur with a diameter of 232 km.
It orbits the Sun between Saturn and Uranus, grazing the orbit of
Uranus. Chariklo has two ice rings (named Oiapoque and Chu�) with
radii 396 and 405 km and widths of about 7 km and 3.5 km respectively.
\item [{10370~Hylonome}] A minor planet orbiting in the outer Solar System.
The dark and icy body belongs to the class of centaurs and measures
approximately 75 kilometers in diameter. It is a Neptune-crosser,
and an outer-grazer of the orbit of Uranus, which it hence does not
cross.
\end{description}

\section{Communication Delay}

One of the few true constants in the universe remains the speed of
light. Whether communications are sent as radio waves or on the laser
of a tightbeam, they travel 300,000 kilometers per second. Often a
message can be slowed while the receiver waits for redundant copies
of lost packets of data to arrive, or for a message to work its way
to the top of the queue at a tightbeam relay station and get passed
along on the next stage of its journey. Worse, a distance of 15 light
minutes between Earth and Ceres doesn\textquoteright t mean it takes
15 minutes to establish a connection: it means it takes 15 minutes
for the first part of the message to travel the distance (\textquotedblleft How
are you?\textquotedblright ), then another 15 minutes for the reply
to return (\textquotedblleft I\textquoteright m fine, thanks.\textquotedblright ).
Conversations of alternating messages can take hours or days to complete.
Because of this, most transmitted conversations are sent as recorded
messages rather than attempting a live conversation, unless the light-speed
delay is only a few seconds.

\begin{figure}[h]
\includegraphics[width=1\columnwidth]{output/delay}
\end{figure}

Communication delay between planet's moons is usually less than 10
seconds. Delay between Earth and Moon is about 1.3 seconds.

\section{Travel Times}

Because the Epstein Drive removed \ensuremath{\Delta}v limitations
to space travel, ships no longer needed to execute Hohmann transfers
to maneuver between bodies. To transfer, ships can simply burn prograde
at a constant rate, then flip around and decelerate by burning retrograde.
This maneuver is called a Brachistochrone trajectory, from the Greek
meaning \textquotedblleft shortest time.\textquotedblright{} A course
using this maneuver tends to be curved as it uses the sun\textquoteright s
gravity to increase the ship\textquoteright s acceleration.

The time it takes to execute a flight path along a Brachistochrone
trajectory depends on two factors: the distance between the two points,
and the acceleration of the ship. The greater the acceleration, the
less time it takes to travel, but the multiple g-forces created by
these \textquotedblleft hard burns\textquotedblright{} are extremely
stressful on the human body. Crash couches and pharmaceutical cocktails
like \textquotedblleft the juice\textquotedblright{} alleviate some,
but not all, of the damage inflicted by the hardest burns.

The following tables show the time to travel the average distance
between the two locations at an average acceleration of 0.5g and 1.0g,
presented as total required travel time under acceleration without
any of the necessary breaks taken into consideration. The first one
is tolerable for belters and the second one suitable for earthers.

\begin{figure}[h]
\includegraphics[width=1\columnwidth]{output/travel05}
\end{figure}

\begin{figure}[h]
\includegraphics[width=1\columnwidth]{output/travel10}
\end{figure}

The last table starts and ends journey with acceleration boost. The
central part of the journey uses normal 0.5g acceleration and flip
at the middle.

\begin{figure}[h]
\includegraphics[width=1\columnwidth]{output/travel05+boost}
\end{figure}

\end{document}
